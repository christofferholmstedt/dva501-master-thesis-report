\chapter{Designing and modeling the system}\label{ch:uppaal_models}
This chapter includes the UPPAAL models in conjunction with corresponding UML
diagrams describing the implementation done in Ada. The UPPAAL models are used
for the validation and verification of the system that it meets the real-time
requirements set on the system. The connection from the UPPAAL models to the
UML diagrams presented here are to show that the implementation corresponds as
close as possible to the models verified.

\section{UPPAAL models}
Each SPA application will do the same thing receive, handle and send SPA
messages. The difference between the applications is which messages each
application understands and which messages the application will ignore.
Therefore the UPPAAL models have been created so an application model can be
instantiated with the specific message types it handles in separate automata.
Each SPA application is modelled on its own to reduce complexity. A driver
model is used to vary the amount of incoming messages.

\subsection{A template application}
TODO: What is common between all applications.

\subsection{Central Addressing Service (CAS)}
TODO: Add models for the specific messages this one handles.

\subsection{Local Subnet Manager (SM-L)}
TODO: Add models for the specific messages this one handles.

\section{Validation and Verification}
TODO: Add comments about time verification and correctness.
