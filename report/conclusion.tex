\chapter{Conclusion}\label{ch:conclusion}
TODO: What does this work bring? What is new and what is good about it?

TODO: What are my conclusion(s)?

TODO: Is both the event- and time-triggered systems verified correcly with the
models and are the models good abstractions from the real-world implementation?
Is event- or time-triggered systems best in this field?

TODO: Trade off between detail in models and complexity in models. Higher
detail means higher complexity. Is the current abstraction a good trade-off?

\section{Future research and development}
TODO, discussion \#1: Is this approach worth going forward with. More input is
needed regarding future operations. The presented solution which uses Ada
Protected Object requires future updates to the software in a processing node
to replace the entire binary running. There is no possibility for dynamic
linking of modules.  If there processing node has an external "manager" that
can supervise upgrades and verify that they have been installed correctly it
may be worth it, though for other embedded systems which run alone may not have
the luxury of a supervisor and an upgrade of the entire binary may be
problematic.

TODO, discussion \#2: Security. VERY important if you aim for the IoT market,
isolated robots and satellites might be a different thing but devies connected
to the internet must be secured. Security. Proof of concept has been shown and
now security must be added as early as possible so it doesn't have to be glued
on top of a system later on e.g. in a home automation, what restricts
neighbouring systems (different apartments) to talk to each other as the link
layer often is the same within a building.

TODO: Routing through several subnets with different speeds. Maximum
Transmission Unit and routing through the networks with the best throughput
should be looked into.

TODO: What should be extended with regard to VN and what could be improved with
regard to the implementation?

TODO: To improve testing of components a IPSEC/gateway for remote location
communication could be added. Imagine that you have a development team in USA,
Sweden, South Africa, Japan and China. All development teams work independently
on their own SPA components. It's the responsiblity of the team in Sweden to do
the integration testing so instead of waiting for all others to finish their
product and send it to Sweden (could be both hardware and software), they
connect with a SPA IPSEC gateway to all remote locations and the remote teams
hook up their respective component. The teams have now created a plug-and-play
system over the internet for early integration testing. This could also be used
for remote validation and verification, the verifier connect to a remote site
which have the component to be verified and can thereby test remotely.

TODO: "The Implementation of a Plug-and-play Satellite Bus" writes about a
"Mission Spacecraft Design Tool" for rapid prototyping. This sounds like a
really cool idea. Blender or Inkscape GUI, with satellite/unmanned vehicles/IoT
purpose. The tool could create the models used for verification and the source
code that runs on the different processing nodes.

% \section{All references}
% TODO: Remove this section from your own report when it's done so only the
% references you use are actually added to the bibliography \cite{*}.
