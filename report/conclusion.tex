\chapter{Conclusion}\label{ch:conclusion}
Presented in this thesis report is a model-checked implementation of a new
subnet local adaptation of the Space plug-and-play Architecture. It has been
implemented as the Virtual Network Library \cite{web:github-vn-lib} for the
fork of SPA called "Virtual Network Protocol". Instead of UDP/IP for IPC, Ada
Protected Objects are used to share data between different Ada tasks.

When model-checking a system to verify that it behaves as expected it is
important to get the model to resemble the implementation as close as possible
and vice versa. It is a network of two development artifacts that prove that the
system behaves according to the specifications. In this thesis the
specifications are the SPA standards and the implementation is the "Virtual
Network Library" implementation.

In chapter \ref{ch:designing_and_modeling} the process that took place to reach
the end result regarding the UPPAAL model and the implementation is
shown. Going further with an even more detailed model would go into
unnecessary detail such as how the routing tables or the CAS table are
implemented. As the different tables are not going to hold any substantial
amount of data, the time to do a lookup in the different tables are
negligible. This among other things led to the conclusion that the level of
abstraction of the model is at the right level for the purpose of this thesis.

The verification of the model shows that the only way to reach a deadlock is in
the end state as expected, the problem comes with livelock. The
verification query for livelock shows several different scenarios on how the
system can reach a state of livelock. A very simple example is in the beginning
when an application sends a Local\_Hello to the SM-L and the SM-L receives that
message. When the SM-L is back in idle state the system reaches a livelock
scenario because the application will keep sending Local\_Hello messages
until it receives a Local\_Ack response. Similar scenarios arose throughout
the development of the model, for example when the SM-L sends a request to
the CAS for an Address Block. Unless the CAS responds with an Address Block the
SM-L would keep sending Address\_Block\_Requests forever. The solution to this
scenario was to limit the SM-L to only send one message and make the
assumption that all messages sent will be received. As communication is through
Ada Protected Objects this is an acceptable assumption to make in the current
implementation.

The same solution could be used to the livelock problem with Local\_Hello
messages. The drawback would be that if the application would be moved to any
other type of subnet, the assumption about guaranteed delivery would not hold.
In the end some kind of time based retry algorithm could be used, such as to
resend the Local\_Hello message until a Local\_Ack is received without
resending too often. To be able to model the different SPA applications in more
detail they could be separated out and connected to an automaton representing
the environment.  This would allow testing of specific timings in different
scenarios.

% ...
%
% TODO: Is both the event- and time-triggered systems verified correcly with the
% models and are the models good abstractions from the real-world implementation?
% Is event- or time-triggered systems best in this field?

\subsubsection{Future research - maintainability}
One major consequence of moving away from UDP/IP as IPC is that the
applications running on the different processing nodes no longer are modular in
the sense that it is easy to add and remove applications on a processing node
in a running system. When using Ada Protected Objects all SPA applications must
be compiled together into one binary and started at the same time. This leads
to future updates must replace the entire binary and reboot the system. The
trade-off between Ada Protected Objects and a run-time modularity is something
that need to be prioritized per project. For robots or satellites that are not
connected to internet the increased reliability of protected objects may be a
good trade-off. On the other side devices connected to internet that always
must be up to date security wise to protect against intrusions, the modularity
may be prioritized.  An example to this could be a refrigerator that is
connected to internet and a vulnerability is found in its transport layer
security code. Instead of updating the entire running binary it would be enough
just to update the specific module that exposes the vulnerability. Future work
should take a deeper look at this trade-off, when is it worth the effort to use
Ada protected objects over the modularity of UDP/IP for IPC?

% Future work should look into the possibility of keeping some run-time
% modularity and introducing Ada Protected Objects.

\subsubsection{Future research - security in general and confidentiality in
particular}
For closed-circuit networks like remotely operated vehicles and some robots,
security may not be that important but for the vast majority of devices
security needs to be introduced. As soon as a device is connected to the
Internet and starts to communicate with others it will need some kind of
authentication mechanism to allow access to its own data, it will need
authorization mechanism to be able to differentiate between other devices
regarding which device should get access to what data. Added to this is
transport encryption (confidentiality) to prevent pervasive monitoring and the
possibility of disclosing private information to active listeners
(man-in-the-middle attacks).  How to add security to a SPA network with open
access to internet is an important step forward to make SPA a viable option in
the world of Internet of Things.

\subsubsection{Future development - dependability of the development process}
When starting this thesis Ada was chosen as programming language because of its
track record with software that must be dependable to high-levels. Different
ISO standards and white papers from AdaCore introduce good programming
practices, how to develop good quality software and how to use object-oriented
programming in the field of safety-critical applications. The approach all
these papers use is that the developer and/or architect need to decide upon a
few things first, before starting development.

First of the project must know what kind of safety level is expected from the
system. This relates directly to how much testing, validation and verification
must be done throughout the development cycles and in the end when delivering
the system. Testing, validation and verification
techniques then influences the choice of programming features that can and more
interestingly, should be used.

During the development and design phases in this project quite a lot of
discussions went into which level of safety that was expected. With a specific
product in mind and its requirements this would have been less time consuming.
In the end, without specific safety requirements focus was put on which Ada
features that were thought to be useful and which ones should be avoided. The
reasoning behind this is that the fewer Ada constructs that were used the
easier it would be to validate and verify in the future (and porting the system
to new architectures).  Future work should take a deeper look at the different
safety standards that may be of interest and come up with firm programming
guidelines that can be used during development.

% TODO: Routing through several subnets with different speeds. Maximum
% Transmission Unit and routing through the networks with the best throughput
% should be looked into.
%
% TODO: What should be extended with regard to VN and what could be improved with
% regard to the implementation?
% 
% TODO: To improve testing of components a IPSEC/gateway for remote location
% communication could be added. Imagine that you have a development team in USA,
% Sweden, South Africa, Japan and China. All development teams work independently
% on their own SPA components. It's the responsiblity of the team in Sweden to do
% the integration testing so instead of waiting for all others to finish their
% product and send it to Sweden (could be both hardware and software), they
% connect with a SPA IPSEC gateway to all remote locations and the remote teams
% hook up their respective component. The teams have now created a plug-and-play
% system over the internet for early integration testing. This could also be used
% for remote validation and verification, the verifier connect to a remote site
% which have the component to be verified and can thereby test remotely.
% 
% TODO: "The Implementation of a Plug-and-play Satellite Bus" writes about a
% "Mission Spacecraft Design Tool" for rapid prototyping. This sounds like a
% really cool idea. Blender or Inkscape GUI, with satellite/unmanned vehicles/IoT
% purpose. The tool could create the models used for verification and the source
% code that runs on the different processing nodes.

% \section{All references}
% TODO: Remove this section from your own report when it's done so only the
% references you use are actually added to the bibliography \cite{*}.
