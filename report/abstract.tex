\chapter*{Abstract}
\thispagestyle{empty} % No page numbering on this page
Space Plug-and-Play Architecture (SPA) is a set of standards to make it easier
to build small satellites. Focus is put on improving the integration phase and
the time consuming validation and verification process by introducing
plug-and-play functionality. From mission call-up to operational satellite it
should only take six days.

A SPA network consists of several different types of subnets with different
pros and cons. For each processing node there must be one Local Subnet Manager
(SM-L). The SM-L can communicate over different communication protocols
depending on how the respective local subnet is set up, one option is UDP/IP.

In this thesis Ada Protected Objects is presented as a viable option for
inter-process communication instead of UDP/IP in a SPA network. This thesis
presents the initial work towards a SPA Local Subnet Adaptation that builds on
language constructs in Ada such as Ada Tasks and Protected Objects.  The system
design and implementation is verified deadlock free with UPPAAL but shows
indications of livelock possibilities. The severity of these livelock
situations is discussed in the conclusion.

% Keywords
\textbf{Keywords:} Space plug-and-play Architecture, SPA, UPPAAL, Timed
automata, Ada, Ravenscar, Unified Common Architecture, UNICA, Virtual Network
Protocol, VNP
