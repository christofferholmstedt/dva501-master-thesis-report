\chapter{State of the Art} \label{ch:state_of_the_art}

\section{Space plug-and-play Architecture}
The Air Force Resarch Laboratory (AFRL) Space Vehicle Directorate started a
project referred to as "Space plug-and-play Avionics" in 2004 with the intent
to understand why space systems used in various mission became so complex and
how they can be made easier to assembly \cite{fronterhouse2007, martin2012}.
The solution was to look at plug-and-play systems already available on ground.
An example to this is how USB devices work with laptop and desktop computers in
everyday life. It doesn't matter which keyboard you buy the basic functionality
will work in any computer you plug it into, no matter if it runs Windows, OS X
or a Linux distribution.

"Space plug-and-play Avionics" was released during the 2000-decade as a set of
standards and in an updated version from 2011 renamed to "Space plug-and-play
Architecture" \cite{martin2012}, all standards are available from AIAA
\cite{spa:all}.

\nomenclature{\textbf{AFRL}}{\textbf{Air Force Resarch Laboratory}}

Development and testing of implementations conforming to the standards have
been ongoing since the launch of the program.

\section{Dependability and Ada}
TODO: Start from the point of view of "Dependability" and which parts are in
focus in this thesis. Finish of with a few notes about Ada and its use in
high-integrity systems.

\section{Model Checking and UPPAAL}
TODO: Start with model checking in general, introduce the concept of timed
finite automata and give examples on how it has been used by others.

TODO: Some notes about UPPAAL and the different forks available.

\section{Internet of Things}
TODO: Do a broad overview of the concept of Internet of Things and the
different standards that are available.
