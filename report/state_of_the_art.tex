\chapter{State of the Art} \label{ch:state_of_the_art}

\section{Space plug-and-play Avionics}
\subsection{Basic concepts and terminology in SPA}
Space plug-and-play Avionics (SPA) is a set of standards that define how
different players can create plug-and-play components that can easily be
assembled to form a complete satellite for a specific mission. The standards
are published by American Institute of Aeronatuics and Astronautics (AIAA). The
standards range from how power supply should be handled to how components
communicate in the application layer, focus in this report is put on the
software parts.

To describe how a SPA network works some terminology needs to be defined. All
nodes in a SPA network are called "components", there is no differentiation
between software and hardware components. Each component has a "Component
Universally Unique Identifier" (CUUID) which is 128 bit long. Each component
also has a "Extensible Transducer Electronic Data Sheet" (xTEDS) file that
describes respective components capabilities. To be able to support different
underlying link layer technologies such as Ethernet, Spacewire and one-wire
(I2C), most functionality has been put in the application layer. This means
that a SPA network can be built with a combination of link layer "subnets". As
an example a SPA network can consist of two Spacewire subnets and one Ethernet
subnet.

The important concepts to remember is that each SPA component is
connected to one or more SPA subnets, on each SPA subnet there is a subnet
manager (SM-x) that acts as a gateway to other SPA subnets and when SPA subnets
are linked together they form a SPA network.

Another SPA subnet is the SPA Local Subnet. Each processing node that can
have multiple SPA components running on it has a "SPA Local Subnet Manager"
(SM-L).  Components within a processing node do inter-process communication
over UDP/IP.

Components communicate with each other with the help of "logical addresses"
that each component recieves during boot up. It's the responsiblity of the
"Central Addressing Service" (CAS) to hand out logical addresses to all
components in the SPA network. After a component receives its logical address
it can share its xTEDS file with other components. It's the responsibility of
the "Lookup Service" to keep track of all components' capabilities in the SPA
network.

To describe what a SPA network can look like figure TODO shows a minimal SPA
network which consists of a CAS, LS, a sensor and a monitor, all running on the
same processing node and in the same local subnet managed by the SM-L.

A more complex SPA network is shown in figure TODO. The original SPA network is
extended with multiple SPA subnets connected to the original SPA Local Subnet.

For a detailed view on how SPA works the SPA standards is the best source for
information though chapter \ref{ch:spa} will go through some scenarios in
detail that can be helpful to the reader to understand SPA. The SPA standards
and drafts that specify software requirements are the Logical Interface
standard \cite{spa:logical-interface}, Networking standard
\cite{spa:networking}, Local Subnet Draft (TODO: Ref), Ontology Standard
\cite{spa:ontology} and System Capabilities Standard
\cite{spa:system-capabilities}.

\subsection{Related research on SPA}
TODO: Find case studies and other research articles that relates to SPA and how
it works in practice.

\subsection{Virtual Network and the Virtual Network Protocol}
TODO: Finish of with the defintion of Virtual Network, Virtual Network
Protocol and how it relates to SPA.

\section{Dependability and Ada}
TODO: Start from the point of view of "Dependability" and which parts are in
focus in this thesis. Finish of with a few notes about Ada and its use in
high-integrity systems.

\section{Model Checking and UPPAAL}
TODO: Start with model checking in general, introduce the concept of timed
finite automata and give examples on how it has been used by others.

TODO: Some notes about UPPAAL and the different forks available.

\section{Internet of Things}
TODO: Do a broad overview of the concept of Internet of Things and the
different standards that are available.
