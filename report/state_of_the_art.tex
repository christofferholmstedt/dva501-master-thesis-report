\chapter{State of the Art} \label{ch:state_of_the_art}
This chapter on the state of the art in related research fields aims at giving
the reader a basic stepping stone for later chapters where more details are
described concerning Space plug-and-play Architecture and UPPAAL modeling.

\section{Space plug-and-play Architecture}
The Air Force Resarch Laboratory (AFRL) Space Vehicle Directorate started a
project referred to as "Space plug-and-play Avionics" in 2004 with the intent
to understand why space systems used in various missions became so complex and
how they can be made easier to assembly \cite{fronterhouse2007, martin2012}.
The solution was to look at plug-and-play systems already available on ground.
An example to this is how USB devices work with laptop and desktop computers in
everyday life. It doesn't matter which keyboard you buy the basic functionality
will work in any computer you plug it into, no matter if it runs Windows, OS X
or a Linux distribution.

"Space plug-and-play Avionics" was released during the 2000-decade as a set of
standards and in an updated version in 2011 renamed to "Space plug-and-play
Architecture" \cite{martin2012}, all standards are available from AIAA
\cite{spa:all}.

\nomenclature{\textbf{AFRL}}{\textbf{Air Force Resarch Laboratory}}

Development and testing of implementations conforming to the standards have
been ongoing since the launch of the program. One project that has been
presented is the PnPSat by Fronterhouse et al. \cite{fronterhouse2007} and
Martin et al. \cite{martin2008}. They have demonstrated the first created
plug-and-play satellite and given their view on the problems encountered during
development of both hardware and software components. The first plug-and-play
satellite flown on orbit is presented by Martin et al. \cite{martin2012} where
most parts of the plug-and-play experiement passed all tests one year after
launch.

As a critical part of plug-and-play satellite research and development is the
"System Data Model" (SDM) middleware software presented by Sundberg et al.
\cite{sundberg2006}, previously under the name "Satellite Data Model"
\cite{spa:sdm-source}. The SDM can be viewed in two different ways, as a
middleware or as a "sideware" \cite{fronterhouse2007}, depending on how you
view it. The SDM is one implementation of the SPA standards that manage the
parts of network and data capabilities discovery. The SDM source code has been
released under public domain license \cite{spa:sdm-source}.

\nomenclature{\textbf{SDM}}{\textbf{System Data Model}}

To enable components to be self-describing all SPA components must have an
"Extensible Transducer Electronic Data Sheet" (xTEDS) file stored within the
component. The xTEDS file include all the information other components need to
understand its capabilities e.g. which interfaces the component has. The
problem at this point is that multiple vendors start to create their own
interface definitions and all benefits with self-describing components are
lost. This is where the "Common Data Dictionary" (CDD) comes into play.

\nomenclature{\textbf{xTEDS}}{\textbf{Extensible Transducer Electronic Data Sheet}}
\nomenclature{\textbf{CDD}}{\textbf{Common Data Dictionary}}

The first work to create the CDD is described as an informal approach by Pasko.
Instead Pasko suggests a more formal approach using RDF. Pasko suggests that
RDF and RDF Schemas can be used to not only desribe the CDD formally but also
be used to visually represent the CDD with suitable software \cite{pasko2011}.

Hansen et al. \cite{hansen2012} describes the dictionary as an ontology which
consists of several taxonomies in a hierarchial structure. This means that if
different vendors respectively create a SPA component with a temperature sensor
and a pressure sensor interface the vendors use the CDD to describe the
interfaces with the correct vocabulary. To manage the risk of adding too
many terms in the CDD Lanza et al. \cite{lanza2010} suggests the forming of a
CDD panel to "[...] serve as a regulatory structure for the CDD". To make it
easier for developers to write xTEDS files with the correct vocabulary and
improve the CDD, Lanza et al.  \cite{lanza2010} have created a web based tool
to manage the CDD.

One benefit of using XML file structure for the xTEDS is that a XML schema
definition (XSD) can be created and with the help of a XML validating parser
xTEDS can be verified automatically to conform to the xTEDS file format and the
CDD \cite{lanza2010}.

\nomenclature{\textbf{XSD}}{\textbf{XML Schema Definition}}

\section{UPPAAL}
To verify a system meets safety-critical standards different approaches can
be taken and depending on the severity of failures that can be caused by the
system different approaches must be taken. To verify that the core SPA system
implementation presented in this paper meets some basic requirements it will be
modeled and verified with the model-checking tool UPPAAL.

On the official website UPPAAL is described as follows, "UPPAAL is an
integrated tool environment for modeling, validation and verification of
real-time systems modeled as networks of timed automata, extended with data
types (bounded integers, arrays, etc.)." \cite{web:uppaal}. It is maintained
and developed by participants mainly from Uppsala University in Sweden and
Aalborg University in Denmark.

The formal methods behind UPPAAL builds on the research originally presented
by Alur et al. \cite{alur1994} where real-time systems are shown to be modeled
with timed finite automata, or just timed automata in short. The key part is
that the system has a finite amount of states so it's possible to model it in a
network of automata. Since the presentation of the formal method, tools have
been developed to make the method useful. UPPAAL has over the last decade been
forked to more specific versions such as "TIMES" which can be used for
scheduability analysis and "TRON" for online real-time testing, other forks
exist and for a more detailed history of the development process "Developing
UPPAAL over 15 years" by Behrmann et al.  \cite{behrmann2011} is suggested. In
this thesis "classic" UPPAAL is used to verify real-time properties of the
system.

UPPAAL is used mainly to verify real-time systems in a wide variety of research
fields. From the field of radiation therapy \cite{man2011} and safety-critical
systems in the nuclear engineering domain \cite{lahtinen2012} to embedded
system in a university satelite \cite{alencar2013} and flight-program in an
unmanned helicopter \cite{lee2011}. Worth noting is that UPPAAL can also be used
to verify protocols such as in a distributed lift system with a CAN Bus
presented by Pang et el. \cite{pang2003} and the spacewire architecture
presented by Ermont et al. \cite{ermont2013}.

The implementation presented in this thesis is done in Ada 2005 with the
Ravenscar restriction to make it easier to verify real-time requirements in a
model-checking tool. Related to this is earlier work done in UPPAAL regarding
Ada software and tasking. Bj\"{o}rnfot presented "Ada and timed automata" in
1996 \cite{bjornfot1996} and Lundqvist et al. presented verification of an Ada
RTS kernel with the Ravenscar restriction applied in four papers
\cite{lundqvist1999f,lundqvist1999g,lundqvist1999h,lundqvist2003}.

\section{Internet of Things} \label{sec:sota:internet_of_things}
As part of this thesis work a first look at the field of Internet of Things is
included and more specifically the different application layer protocols
available. This section lists the main other protocols and frameworks available
for use in the field of "Internet of Things" which could be related to the
future of SPA and the "Virtual Network Protocol" fork.

\subsubsection{Constrained Application Protocol (CoAP)}
Constrained Application Protocol (CoAP) is an application layer protocol
going through standardization process in the IETF working group "core". The
publication of the CoAP draft as a proposed standard is imminent as the
document has reached "Authors' Final Review" as of May 2014 \cite{web:wgcore}.

CoAP is designed for highly constrained nodes, that is nodes with a 8-bit
microcontroller and very little ROM and RAM. Added to the hardware constrains
the microcontrollers are often connected to lossy networks with low bandwidth.
The protocol is a request-response protocol that relatively easily can be
proxied to HTTP as it works with URIs in the same manner \cite{web:draftcoap}.

\subsubsection{MQ Telemetry Transport (MQTT)}
Originally invented in 1999 MQTT is a publish and subscribe, lightweight
protocol designed for highly constrained nodes, just like CoAP. It runs
over TCP/IP so it expects that network stack on all devices. A version of MQTT
called MQTT-SN for "Sensor Networks" is a branch of the protocol dedicated for
networks without TCP/IP capabilities \cite{web:mqtt,web:mqtt-sn}. The main
protocol is undergoing standardization process in the OASIS organization, MQTT
Working Group \cite{web:oasis-mqtt}.

\subsubsection{AllSeen Open Source project}
From the begining developed by Qualcomm as the AllJoyn project, the AllSeen
Open Source Project is now run as a Linux Foundation Collaborative Project.
It is a framework which offers service discovery, remote procedure calls,
security and a lot of other features independent of the transport layer used.
It is a framework with a core written in C++ with bindings to multiple other
languages and runs on a wide variety of platforms from embedded to desktop
devices \cite{web:alljoyn,web:allseen}.

\subsubsection{BACnet}
In contrast to CoAP, MQTT and AllSeen Open Source Project the goal and primary
concern for BACnet is building automation. BACnet, that is \textbf{b}uilding
\textbf{a}utomation and \textbf{c}ontrol \textbf{net}works, was originally
started as standardization project within the American Society of Heating,
Refrigerating and Air-Conditioning Enerineers (ASHRAE). After publication of
the american standard it has since been published as both an european standard
and an ISO standard \cite{web:bacnet}.

As with SPA, BACnet goes into detail about everything from hardware to
software \cite{web:bacnet-slide15}.
