\chapter{State of the Art} \label{ch:state_of_the_art}

\section{Space plug-and-play Architecture}
The Air Force Resarch Laboratory (AFRL) Space Vehicle Directorate started a
project referred to as "Space plug-and-play Avionics" in 2004 with the intent
to understand why space systems used in various mission became so complex and
how they can be made easier to assembly \cite{fronterhouse2007, martin2012}.
The solution was to look at plug-and-play systems already available on ground.
An example to this is how USB devices work with laptop and desktop computers in
everyday life. It doesn't matter which keyboard you buy the basic functionality
will work in any computer you plug it into, no matter if it runs Windows, OS X
or a Linux distribution.

"Space plug-and-play Avionics" was released during the 2000-decade as a set of
standards and in an updated version in 2011 renamed to "Space plug-and-play
Architecture" \cite{martin2012}, all standards are available from AIAA
\cite{spa:all}.

\nomenclature{\textbf{AFRL}}{\textbf{Air Force Resarch Laboratory}}

Development and testing of implementations conforming to the standards have
been ongoing since the launch of the program. One project that has been
presented is the PnPSat by Fronterhouse et. al. \cite{fronterhouse2007} and
Martin et. al. \cite{martin2008}. They have demonstrated the first created
plug-and-play satellite and given their view on the problems encountered during
development of both hardware and software. The first plug-and-play satellite
flown on orbit is presented by Martin et. al. \cite{martin2012} where most
parts of the plug-and-play experiement passed all tests one year after launch.

As a critical part of plug-and-play satellite research and development is the
"System Data Model" (SDM) middleware software presented by Sundberg et. al.
\cite{sundberg2006}, previously under the name "Satellite Data Model"
\cite{spa:sdm-source}. The SDM can be viewed in two different ways, as a
middleware or as a "sideware" \cite{fronterhouse2007}, depending on how you
view it. The SDM is one implementation of the SPA standards that manage the
parts of handing out logical addresses to all SPA components, doing network and
data capabilities discovery. The SDM source code has been released under public
domain license \cite{spa:sdm-source}.

\nomenclature{\textbf{SDM}}{\textbf{System Data Model}}

To enable components to be self-describing all SPA components must have a xTEDS
file stored within the component. The xTEDS file include all the information
other components need to understand its capabilities e.g. which interfaces the
component have. The problem at this point arise that multiple vendors start to
create their own interface definitions and all benefits with self-describing
components are lost. This is where the "Common Data Dictionary" (CDD) comes
into play.

Hansen et. al.  \cite{hansen2012} describes the dictionary as an ontology which
consists of several taxonomies in a hierarchial structure. This means that if a
vendors creates a SPA component with a temperature sensor and a pressure sensor
interface the vendor use the CDD to describe the interfaces with the correct
vocabulary. To manage the risk of adding way too many terms in the CDD Lanza
et. al. \cite{lanza2010} suggests the forming of a CDD panel to "[...] serve as
a regulatory structure for the CDD". To make it easier for developers to write
xTEDS files with the correct vocabulary and improve the CDD Lanza et. al.
\cite{lanza2010} have created a web based tool to manage the CDD.

One benefit of using XML file structure for the xTEDS is that a XML schema
definition (XSD) can be created and with the help of a XML validating parser
xTEDS can be verified automatically to conform to the xTEDS file format and the
CDD \cite{lanza2010}.

The first work to create the CDD is described as an informal approach by Pasko
\cite{pasko2011}, instead a more formal approach using RDF is suggested. Pasko
suggests that RDF and RDF Schemas can be used to not only desribe the CDD
formally but also be used to visually represent the CDD with suitable software.

\nomenclature{\textbf{XSD}}{\textbf{XML Schema Definition}}
\nomenclature{\textbf{CDD}}{\textbf{Common Data Dictionary}}

\section{Model Checking and UPPAAL}
TODO: Start with model checking in general, introduce the concept of timed
finite automata and give examples on how it has been used by others.

TODO: Some notes about UPPAAL and the different forks available.

\section{Internet of Things}
TODO: Do a broad overview of the concept of Internet of Things and the
different standards that are available.
