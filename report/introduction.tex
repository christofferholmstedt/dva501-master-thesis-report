\chapter{Introduction}

\section{Background}
% TODO: This section should be a summary of the State of the Art, more or less.

% TODO: Fix references in this introduction.

Space plug-and-play Architecture (SPA) is intended to decrease the time needed in
the integration phase when constructing satellites for specific missions. This
is done by introducing plug-and-play interfaces between different components,
both software and hardware components, just like when you plug in a USB mouse
or keyboard in your laptop or desktop computer. SPA is a set of standards
published by American Institute of Aeronatuics and Astronautics (AIAA). The
different standard documents define everything from power supply interfaces to
how components automatically describe themselves to other components.

\nomenclature{\textbf{SPA}}{\textbf{Space Plug-and-Play Architecture} is a set
of standards for plug-and-play technology used in the space industry.
Previously known as "Space plug-and-play Avionics"}
\nomenclature{\textbf{AIAA}}{\textbf{American Institute of Aeronatuics and
Astronautics}}

SPA has been developed since 2004 and has been shown to work as intended. As of
early 2014 efforts are now taking place to broaden the scope of SPA under the
names of "Unified Common Architecture" (UNICA) and "Virtual Network Protocal"
(VNP). The goal with Unified Common Architecture is to introduce SPA concepts
and ideas to the field of robotics, unmanned vehicles, industrial automation
and internet of things. This work is done in a collaboration between
M\"{a}lardalen University (MDH) and Bruhnspace Advanced Projects AB (BAP) in
Sweden.

\nomenclature{\textbf{UNICA}}{\textbf{Unified Common Architecture}}
\nomenclature{\textbf{VNP}}{\textbf{Virtual Network Protocol}}
\nomenclature{\textbf{MDH}}{\textbf{M\"{a}lardalen University}}
\nomenclature{\textbf{BAP}}{\textbf{Bruhnspace Advanced Projects AB}}

The Virtual Network Protocal is a subset of the SPA standards that
strictly limit the focus to the software parts of SPA. The goal is to try to
stay interoperable with SPA as much as possible though some differences might
be necassary to introduce.

As a part of SPA is the inter-process communication (IPC)
through UDP/IP within respective processing node/unit. The dependability of
UDP/IP can be discussed but the starting point for this thesis work is that
UDP/IP is not good enough, better options exist.

\nomenclature{\textbf{IPC}}{\textbf{Inter-Process Communication}}
\nomenclature{\textbf{UDP}}{\textbf{User Datagram Protocol}}
\nomenclature{\textbf{IP}}{\textbf{Internet Protocol}}

To increase dependability of a SPA network this thesis work will look at the
possibilities of developing core parts of SPA in the Ada programming language
which has a good track record in the field of safety-critical systems.
During implementation the system will be modeled with UPPAAL to verify
specified real-time attributes.

% TODO: Is it worth aiming for all five attributes in one thesis?
Before describing the problem statement it's important to define a few key
terms. Both the term safety-critical system and high-integrity system are
commonly used in different fields of research and communities. In this report
the term safety-critical system will be used and the definition of a
safety-critical system is that it's dependable.  "Dependability" is a broad
term and in this thesis the definition used by Avi\v{z}ienis et. al.
\cite{avizienis2004} and Laprie \cite{laprie2008} is used. More specfically the
attributes in focus are availability, reliability, safety, integrity,
maintainability and resilience in no specific order.  Confidentially is deemed
out of scope and left for future work.

\section{Problem Statement}
This thesis work will focus on the core parts of a SPA network which runs on
one or several processing nodes. This includes the Central Addressing Service
(CAS), the Lookup Service (LS) and the Subnet Manager for "Local Subnets"
(SM-L) components. Each component will be implemented as one or several Ada
Tasks and communication between the Ada Tasks will be through Ada Protected
Objects which is a shared memory construct defined as a part of the Ada
programming language.

The different SPA software components (applications) are modelled and
implemented in a modular maintainable design so each application can easily
be switch for another or moved to another processing node.

The following list includes the questions to be answered
\begin{enumerate}
    \item Can any of the core SPA applications reach a state of deadlock?
    \item Can any of the core SPA applications reach a state of livelock?
\end{enumerate}

%%%%%%
% Below is the old definition of the problem statement.
% TODO: Definition of the problem statement(s).

% TODO: Define the different properties/attributes to be measured when
% comparing event-triggered and time-triggered systems.

% TODO: Define what is in scope and what is out of scope.

%This thesis work will focus on the core parts of a SPA network which runs on
%one or several processing nodes. This includes the Central Addressing Service
%(CAS), the Lookup Service (LS) and the Subnet Manager for "Local Subnets"
%(SM-L) components. Each component will be implemented as one or several Ada
%Tasks and communication between the Ada Tasks will be through Ada Protected
%Objects which is a shared memory construct defined as a part of the Ada
%programming language.
%
%For SPA interoperability the SM-L will have two interfaces to other SPA
%components one to other local Ada Tasks through dedicated protected objects and
%one through UDP/IP. The interface to external systems, the UDP/IP interface,
%will be the system boundary of the analyzed system in this thesis.
%
%\nomenclature{\textbf{CAS}}{\textbf{Central Addressing Service} is responsible
%for providing logical address blocks to be assigned to each hardware or
%software component. The CAS stores the logical address block and logical
%address for each SPA Manager in the SPA Network.}
%\nomenclature{\textbf{LS}}{SPA \textbf{Lookup Service}}
%\nomenclature{\textbf{SM-Local}}{\textbf{SPA Local Subnet Manager}}
%
%The following list includes the questions to be answered
%\begin{enumerate}
%    \item Does the system have any possibility to reach a state of deadlock?
%     \item Does the system have any possibility to starve any queue in the
%         system?
%     \item Is a time-triggered or event-triggered system the best option when it comes to
%         \begin{enumerate}
%             \item Worst-case response time from the time a message arrives at
%                 the UDP/IP interface and when a response leaves (this is
%                 measured with a predefined set of message types that will give
%                 a response to 100\% of the requests).
%             % TODO: More criteria?
%         \end{enumerate}
%\end{enumerate}

\section{Structure of this report}
The report is structured as follows. Chapter \ref{ch:state_of_the_art} goes
through the state of the art in Space plug-and-play Architecture, UPPAAL
modeling and verification as well as Internet of Things.  Chapter
\ref{ch:method} continues with the research and development
method used throughout the thesis work, from literature study to modeling and
implementation. A thorough presentation on how Space plug-and-play Architecture 
works is then given in chapter \ref{ch:spa}. Chapter \ref{ch:uppaal_models}
presents all the UPPAAL models used to verify the system, chapter
\ref{ch:result} shows all the results aquired and chapter \ref{ch:conclusion}
finish the report with the discussion about the results, the conclusion and
suggestions for future research and development.

% TODO: Add one paragraph with information on how the following chapters and
% sections are structured. This section title might not be needed for this
% paragraph.
