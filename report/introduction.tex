\chapter{Introduction}

\section{Background}
TODO: This section should be a summary of the State of the Art, more or less.

TODO: Fix references in this introduction.

Space plug-and-play Avionics (SPA) is intended to decrease the time needed in
the integration phase when constructing satellites for specific missions. This
is done by introducing plug-and-play interfaces between different components,
both software and hardware components, just like when you plug in a USB mouse
or keyboard in your laptop or desktop computer. SPA is a set of standards
published by American Institute of Aeronatuics and Astronautics (AIAA). The
different standard documents define everything from power supply interfaces to
how components automatically describe themselves to other components.

\nomenclature{\textbf{SPA}}{\textbf{Space Plug-and-Play Avionics}}
\nomenclature{\textbf{AIAA}}{\textbf{American Institute of Aeronatuics and
Astronautics}}

SPA has been developed since 2004 and has been shown to work as intended. As of
early 2014 efforts are now taking place to broaden the scope of SPA under the
names of "Unified Common Architecture" (UNICA) and "Virtual Network Protocal"
(VNP). The goal with Unified Common Architecture is to introduce SPA concepts
and ideas to the field of robotics, unmanned vehicles, industrial automation
and internet of things. This work is done in a collaboration between
M\"{a}lardalen University (MDH) and Bruhnspace Advanced Projects AB (BAP) in
Sweden.

\nomenclature{\textbf{UNICA}}{\textbf{Unified Common Architecture}}
\nomenclature{\textbf{VNP}}{\textbf{Virtual Network Protocol}}
\nomenclature{\textbf{MDH}}{\textbf{M\"{a}lardalens University}}
\nomenclature{\textbf{BAP}}{\textbf{Bruhnspace Advanced Projects AB}}

The Virtual Network Protocal is a subset of the SPA standards that
strictly limit the focus to the software parts of SPA. The goal is to try to
stay interoperable with SPA as much as possible though some differences may be
necassary to introduce in the future.

As a key part of the SPA standard is the inter-process communication (IPC)
through UDP/IP within respective processing node/unit. The dependability of
UDP/IP can be discussed but the starting point for this thesis work is that
UDP/IP is not good enough, better options exist.

\nomenclature{\textbf{IPC}}{\textbf{Inter-Process Communication}}
\nomenclature{\textbf{UDP}}{\textbf{User Datagram Protocol}}
\nomenclature{\textbf{IP}}{\textbf{Internet Protocol}}

\section{Problem Statement}
TODO: Definition of the problem statement(s).

TODO: Define the different properties/attributes to be measured when comparing
event-triggered and time-triggered systems.

TODO: Define what is in scope and what is out of scope.

\section{Structure of this report}
TODO: Add one paragraph with information on how the following chapters and
sections are structured. This section title might not be needed for this
paragraph.
