\chapter{Method}\label{ch:method}
This chapter describes the method used throughout the thesis work. First
the literature study and then the method used for designing, modeling and
implementation.

\section{Literature study}
To get a general understanding of the field and current state of the art, a
literature study was conducted. The result of the study is presented in chapter
\ref{ch:state_of_the_art}. The literature study for Space plug-and-play
Architecture and the Model Checking was conducted by selecting different key
terms related to the topics and searching for them in two different search
engines. The first one was Google scholar and the second one was the
"Discovery" search engine available to the students and employees of
M\"{a}lardalen University. The Discovery search engine search through multiple
databases such as IEEE Xplore, ACM Digital Library and SpringerLink. For
searches in the Discovery search engine only articles with full text available
was requested.

In Google scholar the first 20 results were looked into and the first 40
results given by Discovery search engine was looked into. Only results with
full text PDFs were selected from Google Scholar.

A first read through of abstracts was conducted to filter out unrelated
papers followed by a read through of introductions and conclusions for
another filtering round. The key terms selected are shown below.

\subsection{Space plug-and-play Architecture}
The key terms chosen were "Space plug-and-play Avionics", "Space
plug-and-play Architecture" and "why Space plug-and-play Avionics".

\subsection{Model Checking and UPPAAL}
The key terms chosen were "UPPAAL", "UPPAAL Modeling" and "UPPAAL Modeling
Ravenscar".

\subsection{Internet of Things}
"Internet of Things" is a broad field for research and development. As this is
only a first look at the topic no specific key terms were selected instead
previously known protocols were selected for a short presentation in section
\ref{sec:sota:internet_of_things}.

\section{An Agile Unified Process}
For the development process inspiration was taken from the Agile Unified
Process presented by Larman in "Applying UML and Patterns" \cite{larman2005}.
As the SPA software is a framework no user stories nor use-cases was created
instead the different SPA standards were used as a feature requirements list.

Throughout the spring regular meetings took place to discuss design issues
with the author of a related thesis work, Nils Brynedal Ignell. Major parts of
the design and implementation have been created in collaboration. The
implementation in Ada with ravenscar restriction in combination with the UPPAAL
model of the system have been created in small increments to reach their final
state.
