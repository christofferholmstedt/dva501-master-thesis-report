\chapter{Method}\label{ch:method}
\section{Literature study}
The literature study for Space plug-and-play Architecture and the Model
Checking was conducted by selecting different key terms related to the topics
and searching for them in two different search engines. The first one was
Google scholar and the second one was the "Discovery" search engine available
for Mälardalen University's students and employees. The Discovery search engine
search through multiple databases such as IEEE Xplore, ACM Digital Library and
SpringerLink. For searches in the Discovery search engine only articles with
full text available was requested.

In Google scholar the first 20 results were looked into and the first 40
results given by Discovery search engine was looked into. Only results with
full text PDFs were selected from Google Scholar.

A first read through of abstracts was conducted to filter out unrelated
papers and then followed by a read through of introductions and conclusions for
another filtering round. The key terms selected are shown below under
respective topic's section.

\subsection{Space plug-and-play Architecture}
The key terms chosen was "Space plug-and-play Avionics", "Space
plug-and-play Architecture" and "why Space plug-and-play Avionics".

All references from two relatively new reports were looked into as well.

% space plug-and-pay avionics
%AFRL plug and play spacecraft avionics.
%developing an ontology.

\subsection{Model Checking and UPPAAL}
TODO: Add key terms.

TODO: Any "key" report that all citations can be looked into?

\subsection{Internet of Things}
"Internet of Things" is a broad field for research and development. In
this thesis work a proof of concept with the Virtual Network implementation is
presented within the area of Internet of Things, more specifically within home
automation. As this is only for the proof of concept no specific key terms were
selected instead previously known protocols were selected for the comparison
presented in section \ref{sec:sota:internet_of_things}.

\section{An Agile Unified Process}
\subsection{Requirements Analysis}
TODO: Write down how I did the requirements analysis and which references I
used for guidance.

\subsection{Design with UML}
TODO: Write down how I did the design and which references I used for guidance.

\subsection{Modeling, validation and verification with UPPAAL}
TODO: Write down how I did the modeling, validation and verification and which
references I used for guidance.

\subsection{Development in Ada}
TODO: Add comments about the ISO standards about safety-critical systems and
the OOP technical report from AdaCore.

TODO: Write about the development in Ada and how regression testing was done.
Test-Driven Development was used and so on.

\subsection{An iterative process}
TODO: Mention that the requirements analysis to development in Ada (sections
above) was iterated to improve the end result.
