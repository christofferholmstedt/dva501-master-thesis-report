\chapter{Ada Programming Language}

%%%%%%%%%%%%%%%%%%%%%%%%%%%%%%%%%%%%%%%%%%%%%%%%
% New reference
%%%%%%%%%%%%%%%%%%%%%%%%%%%%%%%%%%%%%%%%%%%%%%%%
\section{Guide for the use of the Ada programming language in high integrity
systems}

This is a good resource but may be a bit too old in some parts. When using this
as reference make sure do double-check for more up to date information in this
field.

\begin{description}
    \item[Label:] web:ada-high-integrity \cite{web:ada-high-integrity}
    \item[Year:] 2000
\end{description}

ASIS,\nomenclature{\textbf{ASIS}}{\textbf{Ada Semantic
        Interface Specification} provides a standard mechanism for obtaining information
    about an Ada program or its components. ASIS is an ISO standard.}
ACVC,\nomenclature{\textbf{ACVC}}{\textbf{Ada Compiler Validation Capability}
is a set of test which excercises the compiler, linker and the run-time system
together.}
COTS,\nomenclature{\textbf{COTS}}{\textbf{Commercial Off The Shelf}}

\begin{itemize}
    \item Page 3, good list/definition of what "Traceability" is.
    \item Page 3, good definition that "analysis" is static analysis and
        "testing" is dynamic analysis is.
    \item Page 5, good list of verification techniques.
    \item Page 6, section 2.3.9, "Other memory usage" analysis for I/O ports
        could be useful.
    \item Page 11, Analysis of tasks preventing deadlock with. Look into finit
        state automata, petri-nets and model checking.
    \item Page 24, Tagged types should not be returned from subprograms in
        high-integrity systems. What does this mean?
    \item Page 25, section 5.6.2, "notes", item (9), concerning accces types.
        What does this mean?
    \item Page 34, ubounded storage is unacceptable therefore "dynamic
        attributes" are excluded.
    \item Page 34, run-time dispatching is excluded in high-integrity systems.
    \item Page 34, section 5.11 has valuable information about access types and
        dynamic attributes.
    \item Page 34, section 5.11 has valuable information about access types and
    \item Page 35, good reasoning about the use of exceptions.
    \item Page 37, section 5.13 has some good notes on Tasking in
        high-integrity systems though the information might be old. Some notes
        about time-triggered tasks and some about event-triggered tasks.
    \item Page 41, comment about that it's unwise to create a specfic compiler
        for a specific subset of Ada features.
\end{itemize}

%%%%%%%%%%%%%%%%%%%%%%%%%%%%%%%%%%%%%%%%%%%%%%%%
% New reference
%%%%%%%%%%%%%%%%%%%%%%%%%%%%%%%%%%%%%%%%%%%%%%%%
%\section{Title}
%
% Some thoughts about the reference here.
%
%\begin{description}
%    \item[Label:] web:ada-high-integrity \cite{web:ada-high-integrity}
%    \item[Date:] MM-YYYY
%\end{description}
%
%ASIS,\nomenclature{\textbf{ASIS}}{\textbf{Ada Semantic
%        Interface Specification} provides a standard mechanism for obtaining information
%    about an Ada program or its components. ASIS is an ISO standard.}
%
% \begin{itemize}
%     \item Page 34, section 5.11 has valuable information about access types and
%     \item Page 35, good reasoning about the use of exceptions.
%     \item Page 37, section 5.13 has some good notes on Tasking in
% \end{itemize}
