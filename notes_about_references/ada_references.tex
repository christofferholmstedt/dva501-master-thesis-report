\chapter{Ada Programming Language}

%%%%%%%%%%%%%%%%%%%%%%%%%%%%%%%%%%%%%%%%%%%%%%%%
% New reference
%%%%%%%%%%%%%%%%%%%%%%%%%%%%%%%%%%%%%%%%%%%%%%%%
\section{Guide for the use of the Ada programming language in high integrity
systems}

This is a good resource but may be a bit too old in some parts. When using this
as reference make sure do double-check for more up to date information in this
field.

\begin{description}
    \item[Label:] web:ada-high-integrity \cite{web:ada-high-integrity}
    \item[Year:] 2000
    \item[Abbrevations and terms:]
        ASIS,\nomenclature{\textbf{ASIS}}{\textbf{Ada Semantic Interface
            Specification} provides a standard mechanism for obtaining
            information about an Ada program or its components. ASIS is an ISO
            standard.}
        ACVC,\nomenclature{\textbf{ACVC}}{\textbf{Ada Compiler Validation
            Capability} is a set of test which excercises the compiler, linker
            and the run-time system together.}
        COTS,\nomenclature{\textbf{COTS}}{\textbf{Commercial Off The Shelf}}
\end{description}


\begin{itemize}
    \item Page 3, good list/definition of what "Traceability" is.
    \item Page 3, good definition that "analysis" is static analysis and
        "testing" is dynamic analysis is.
    \item Page 5, good list of verification techniques.
    \item Page 6, section 2.3.9, "Other memory usage" analysis for I/O ports
        could be useful.
    \item Page 11, Analysis of tasks preventing deadlock with. Look into finit
        state automata, petri-nets and model checking.
    \item Page 24, Tagged types should not be returned from subprograms in
        high-integrity systems. What does this mean?
    \item Page 25, section 5.6.2, "notes", item (9), concerning accces types.
        What does this mean?
    \item Page 34, ubounded storage is unacceptable therefore "dynamic
        attributes" are excluded.
    \item Page 34, run-time dispatching is excluded in high-integrity systems.
    \item Page 34, section 5.11 has valuable information about access types and
        dynamic attributes.
    \item Page 34, section 5.11 has valuable information about access types and
    \item Page 35, good reasoning about the use of exceptions.
    \item Page 37, section 5.13 has some good notes on Tasking in
        high-integrity systems though the information might be old. Some notes
        about time-triggered tasks and some about event-triggered tasks.
    \item Page 41, comment about that it's unwise to create a specfic compiler
        for a specific subset of Ada features.
\end{itemize}

%%%%%%%%%%%%%%%%%%%%%%%%%%%%%%%%%%%%%%%%%%%%%%%%
% New reference
%%%%%%%%%%%%%%%%%%%%%%%%%%%%%%%%%%%%%%%%%%%%%%%%
\section{Guide for the use of the Ada Ravenscar Profile in high integrity
systems}

This is a good resource. Make sure to double-check if any information is
available and specific to multi-core environments. Has a good list of
abbrevations and terms in the end.

\begin{description}
    \item[Label:] web:ada-ravenscar-high-integrity \cite{web:ada-ravenscar-high-integrity}
    \item[Year:] 2005
    \item[Abbrevations and terms:]
        CSP,\nomenclature{\textbf{CSP}}{\textbf{Communicating Sequential
        Process} A notation for specifying and analyzing concurrent systems.}
        CSS,\nomenclature{\textbf{CSS}}{\textbf{Calculus of Communicating
            Sytems} An algebra for specifying and reasonging about concurrrent
        systems.}
\end{description}


\begin{itemize}
    \item Page iv, Mentions information flow-analysis, schedulability analysis,
        execution-order analysis and model-checking as valuable verification
        methods.
    \item Page 1, 5th paragraph answer the question "Why Ravenscar?".
    \item Page 1, clearly states that Ravenscar has nothing to do with the
        sequential parts of the code.
    \item Page 5-6, section 2.2, Mapping "Ada to the Scheduling Model" is
        interesting. Answers why no task hierarchy is allowed.
    \item Page 7, Ravenscar is mostly about schedulability analysis.
    \item Page 22, add Partition\_Elaboration\_Policy(Sequential) to let
        elaboration finish first in a concurrent program.
    \item Page 26, section 5.5, definition on when protected entries, protected
        procedures and protected functions should be used.
    \item Page 27, section 5.7, some information about suspension objects and
        event triggered systems.
    \item Page 34, section 5.12, dynamic creationand termination of task can be
        simulated.
    \item Page 37, section 6.2, the purpose of static analysis of concurrent
        code.
    \item Page 38, assumption in last paragraph. Analysis techniques must
        support access to volatile data. This is necessary/valid for program
        wide information flow analysis.
    \item Page 40, Pragma\_Elaboration\_Policy(Sequential). Mentions the
        reasoning behind this pragrma.
    \item Page 41, all tasks must have a static priority. Priority Ceiling
        Violation Check.
    \item Page 41-42, task termination can be statically proved abscent.
    \item Page 42-43, Good section about scheduling analysis.
    \item Page 45, Scheduling Analysis calculations. Mentions some
        documentation parts that must be made available by each kernel
        implementations. This is specified in the Real-Time Annex. This is so
        proper overhead for schedulability calculations can be calculated.
    \item Page 50, coding style, mentions HRT-HOOD and how to structure
        packages and a line about naming convention.
    \item Page 64, Activation\_Manager example looks interesting.
\end{itemize}

%%%%%%%%%%%%%%%%%%%%%%%%%%%%%%%%%%%%%%%%%%%%%%%%
% New reference
%%%%%%%%%%%%%%%%%%%%%%%%%%%%%%%%%%%%%%%%%%%%%%%%
%\section{Title}
%
% Some thoughts about the reference here.
%
%\begin{description}
%    \item[Label:] web:ada-high-integrity \cite{web:ada-high-integrity}
%    \item[Date:] MM-YYYY
%    \item[Abbrevations and terms:]
%       ASIS,\nomenclature{\textbf{ASIS}}{\textbf{Ada Semantic Interface
%       Specification} provides a standard mechanism for obtaining information
%       about an Ada program or its components. ASIS is an ISO standard.}
%\end{description}
%
% \begin{itemize}
%     \item Page 34, section 5.11 has valuable information about access types
%     \item Page 35, good reasoning about the use of exceptions.
%     \item Page 37, section 5.13 has some good notes on Tasking in
% \end{itemize}
