\chapter{High Integrity Systems}

%%%%%%%%%%%%%%%%%%%%%%%%%%%%%%%%%%%%%%%%%%%%%%%%%%%%%%%%%%%%%%%%%
% Basic Concepts and Taxonomy of Dependable and Secure Computing
%%%%%%%%%%%%%%%%%%%%%%%%%%%%%%%%%%%%%%%%%%%%%%%%%%%%%%%%%%%%%%%%%
\section{Basic Concepts and Taxonomy of Dependable and Secure Computing}

If you're missing any definition of a word when it comes to dependability this
is a good "root" document to start from. Search for reports that has cited this
one. It starts from the concept of "Dependability and Security".

\begin{description}
    \item[Label:] avizienis2004 \cite{avizienis2004}
    \item[Year:] 2004
    \item[Abbrevations and terms:] None.
\end{description}

The page numbers below refer to the IEEE Transactions on dependable and secure
computing vol. 1, no. 1, January - March 2004 version.
\begin{itemize}
    \item Page 13, section 2.3, mentions the attributes availability,
        reliability, safety, integrity and maintainability.
    \item Page 23, section 4.3, good summary of "security".
    \item Page 27, section 5.3.1, defintiion of static analysis and dynamic
        analysis.
    \item Page 28, section 5.3.1, mentins designing techniques "design for
        verifiabiltiy" (software) and "design for testability" (hardware).
\end{itemize}

%%%%%%%%%%%%%%%%%%%%%%%%%%%%%%%%%%%%%%%%%%%%%%%%%%%%%%%%%%%%%%%%%
% From dependability to resilience
%%%%%%%%%%%%%%%%%%%%%%%%%%%%%%%%%%%%%%%%%%%%%%%%%%%%%%%%%%%%%%%%%
\section{From dependability to resilience}

Defines what "resilience" is.

\begin{description}
    \item[Label:] laprie2008 \cite{laprie2008}
    \item[Year:] 2008
    \item[Abbrevations and terms:] None.
\end{description}

\begin{itemize}
    \item What is the definition of "justified trusted service"? Read up on
        that in Avizienis 2004 \cite{avizienis2004}.
\end{itemize}

%%%%%%%%%%%%%%%%%%%%%%%%%%%%%%%%%%%%%%%%%%%%%%%%
% New reference
%%%%%%%%%%%%%%%%%%%%%%%%%%%%%%%%%%%%%%%%%%%%%%%%
%\section{Title}
%
% Some thoughts about the reference here.
%
%\begin{description}
%    \item[Label:] web:ada-high-integrity \cite{web:ada-high-integrity}
%    \item[Date:] MM-YYYY
%    \item[Abbrevations and terms:]
%       ASIS,\nomenclature{\textbf{ASIS}}{\textbf{Ada Semantic Interface
%       Specification} provides a standard mechanism for obtaining information
%       about an Ada program or its components. ASIS is an ISO standard.}
%\end{description}
%
%
% \begin{itemize}
%     \item Page 34, section 5.11 has valuable information about access types and
%     \item Page 35, good reasoning about the use of exceptions.
%     \item Page 37, section 5.13 has some good notes on Tasking in
% \end{itemize}
