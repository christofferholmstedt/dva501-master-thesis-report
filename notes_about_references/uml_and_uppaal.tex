\chapter{Design with UML and UPPAAL}

%%%%%%%%%%%%%%%%%%%%%%%%%%%%%%%%%%%%%%%%%%%%%%%%%%%%%%%%%%%%%%%%%
% Analyzing the Redesign of a Distributed Lift System in UPPAAL
%%%%%%%%%%%%%%%%%%%%%%%%%%%%%%%%%%%%%%%%%%%%%%%%%%%%%%%%%%%%%%%%%
\section{Analyzing the Redesign of a Distributed Lift System in UPPAAL}

A well explained case study about a lift system with a CAN bus modeled in
UPPAAL to detect some liveness and safety issues. This report might have some
valuable references but it's already 10 years old.

\begin{description}
    \item[Label:] pang2003 \cite{pang2003}
    \item[Year:] 2003
    \item[Abbrevations and terms:] None.
\end{description}

\begin{itemize}
    \item No specifc comments.
\end{itemize}

%%%%%%%%%%%%%%%%%%%%%%%%%%%%%%%%%%%%%%%%%%%%%%%%%%%%%%%%%%%%%%%%
% Object-Oriented Systems Analysis and Design: Using UML
%%%%%%%%%%%%%%%%%%%%%%%%%%%%%%%%%%%%%%%%%%%%%%%%%%%%%%%%%%%%%%%%
\section{Object-Oriented Systems Analysis and Design: Using UML}
A good book which gave me a crash course in UML or rather repitition on which
diagrams to use when and why. A few good examples in the book, didn't read the
case-studies.

\begin{description}
    \item[Label:] book:bennett2010 \cite{book:bennett2010}
    \item[Date:] 2010
    \item[Abbrevations and terms:]
        UML,\nomenclature{\textbf{UML}}{\textbf{Unified Modeling Language}}
        USDP,\nomenclature{\textbf{USDP}}{\textbf{Unified Software Development
        Process}}
\end{description}


\begin{itemize}
    \item Page 118, UML has two types of diagrams, structural and behavioural.
    \item Page 120, This page shows icons for package, subsystem and model.
    \item Page 123, section 5.3.2, Activity diagrams.
    \item Page 127, mentions "activity partitions" and its old name
        "swimlanes".
    \item Page 129, fig 5.15, Figure of the USDP (Unified Software Development
        Process).
    \item Page 141, section 6.2.2, three types of requirements: functional,
        non-functional and usability.
    \item Page 157-158, mentions difference between <<extend>> and <<include>>.
    \item Page 198-201, section 7.5.2, interesting section about boundary,
        entity and control stereotypes.
    \item Page 236-237, section 8.2.3, mentions generalization, encapsulation
        and information hiding. Short paragraph about compoisition.
    \item Page 242, section 8.3.2, composition in detail.
    \item Page 248-251, section 8.4.2, components in detail.
    \item Page 263, Figure 9.3, great sequence diagram example.
    \item Page 264, section 9.3.1, describes how frames are used e.g. in a
        sequence diagram (sd).
    \item Page 278, section 9.3.7, difference between active and passive
        objects in sequence digrams.
    \item Page 279, list of sequence digram's "integration operators" e.g. alt,
        opt and loop.
    \item Page 286, section 9.6, timing diagrams, related to state machines
        (chapter 11).
    \item Page 292, chapter 10. How to specify operations with pre/post
        condtions and contracts.
    \item Page 354, section 12.5.1, list of qualities (quality attributes(?))
        with descriptions.
    \item Page 422, chapter 15 about Design patterns goes through singleton and
        composite.
\end{itemize}

%%%%%%%%%%%%%%%%%%%%%%%%%%%%%%%%%%%%%%%%%%%%%%%%%%%%%%%%%%%%%%%%%%%%%
% A Ravenscar-Compliant Run-Time Kernel for Safety Critical Systems
%%%%%%%%%%%%%%%%%%%%%%%%%%%%%%%%%%%%%%%%%%%%%%%%%%%%%%%%%%%%%%%%%%%%%
\section{A Ravenscar-Compliant Run-Time Kernel for Safety Critical Systems}
A good walkthrough about how a "naked" Run-Time Kernel is verified with UPPAAL.
Includes a good collection of references on Model-checking with automatons.

\begin{description}
    \item[Label:] lundqvist2003 \cite{lundqvist2003}
    \item[Date:] 2003
    \item[Abbrevations and terms:] None.
\end{description}

\begin{itemize}
    \item Page 8, section 4, some interesting references about timed automata
        and model-checking.
\end{itemize}
%%%%%%%%%%%%%%%%%%%%%%%%%%%%%%%%%%%%%%%%%%%%%%%%
% New reference
%%%%%%%%%%%%%%%%%%%%%%%%%%%%%%%%%%%%%%%%%%%%%%%%
%\section{Title}
%
% Some thoughts about the reference here.
%
%\begin{description}
%    \item[Label:] web:ada-high-integrity \cite{web:ada-high-integrity}
%    \item[Date:] MM-YYYY
%    \item[Abbrevations and terms:]
%       ASIS,\nomenclature{\textbf{ASIS}}{\textbf{Ada Semantic Interface
%       Specification} provides a standard mechanism for obtaining information
%       about an Ada program or its components. ASIS is an ISO standard.}
%\end{description}
%
%
% \begin{itemize}
%     \item Page 34, section 5.11 has valuable information about access types and
%     \item Page 35, good reasoning about the use of exceptions.
%     \item Page 37, section 5.13 has some good notes on Tasking in
% \end{itemize}
