\chapter{Design with UML and UPPAAL}

%%%%%%%%%%%%%%%%%%%%%%%%%%%%%%%%%%%%%%%%%%%%%%%%%%%%%%%%%%%%%%%%%
% Analyzing the Redesign of a Distributed Lift System in UPPAAL
%%%%%%%%%%%%%%%%%%%%%%%%%%%%%%%%%%%%%%%%%%%%%%%%%%%%%%%%%%%%%%%%%
\section{Analyzing the Redesign of a Distributed Lift System in UPPAAL}

A well explained case study about a lift system with a CAN bus modeled in
UPPAAL to detect some liveness and safety issues. This report might have some
valuable references but it's already 10 years old.

\begin{description}
    \item[Label:] pang2003 \cite{pang2003}
    \item[Year:] 2003
    \item[Abbrevations and terms:] None.
\end{description}

\begin{itemize}
    \item No specifc comments.
\end{itemize}

%%%%%%%%%%%%%%%%%%%%%%%%%%%%%%%%%%%%%%%%%%%%%%%%%%%%%%%%%%%%%%%%%
% Applying uml and patterns
%%%%%%%%%%%%%%%%%%%%%%%%%%%%%%%%%%%%%%%%%%%%%%%%%%%%%%%%%%%%%%%%%
\section{Applying uml and patterns}


\begin{description}
    \item[Label:] larman2005 \cite{larman2005}
    \item[Year:] 2005
    \item[Abbrevations and terms:] None.
\end{description}

\begin{itemize}
    \item Page 6, section 1.2, the most important skill to learn is "A critical
        ability in OO development is to skillfully assign responsibilities to
        software objects."
    \item Page 30, section 2.7, Adopting an agile method does not mean avioding
        any modeling; that's a misunderstanding.
    \item Page 30, section 2.7, the purpose of modeling and models is primarily
        to support understanding and communication, not documentation.
    \item Page 30, section 2.7, don't model or apply the UML to all or most of
        the software design. Defer simple or straightforward design problems
        until programming - solve them while programming and testing.
\end{itemize}

%%%%%%%%%%%%%%%%%%%%%%%%%%%%%%%%%%%%%%%%%%%%%%%%%%%%%%%%%%%%%%%%
% Object-Oriented Systems Analysis and Design: Using UML
%%%%%%%%%%%%%%%%%%%%%%%%%%%%%%%%%%%%%%%%%%%%%%%%%%%%%%%%%%%%%%%%
\section{Object-Oriented Systems Analysis and Design: Using UML}
A good book which gave me a crash course in UML or rather repitition on which
diagrams to use when and why. A few good examples in the book, didn't read the
case-studies.

\begin{description}
    \item[Label:] book:bennett2010 \cite{book:bennett2010}
    \item[Date:] 2010
    \item[Abbrevations and terms:]
        UML,\nomenclature{\textbf{UML}}{\textbf{Unified Modeling Language}}
        USDP,\nomenclature{\textbf{USDP}}{\textbf{Unified Software Development
        Process}}
\end{description}


\begin{itemize}
    \item Page 118, UML has two types of diagrams, structural and behavioural.
    \item Page 120, This page shows icons for package, subsystem and model.
    \item Page 123, section 5.3.2, Activity diagrams.
    \item Page 127, mentions "activity partitions" and its old name
        "swimlanes".
    \item Page 129, fig 5.15, Figure of the USDP (Unified Software Development
        Process).
    \item Page 141, section 6.2.2, three types of requirements: functional,
        non-functional and usability.
    \item Page 157-158, mentions difference between <<extend>> and <<include>>.
    \item Page 198-201, section 7.5.2, interesting section about boundary,
        entity and control stereotypes.
    \item Page 236-237, section 8.2.3, mentions generalization, encapsulation
        and information hiding. Short paragraph about compoisition.
    \item Page 242, section 8.3.2, composition in detail.
    \item Page 248-251, section 8.4.2, components in detail.
    \item Page 263, Figure 9.3, great sequence diagram example.
    \item Page 264, section 9.3.1, describes how frames are used e.g. in a
        sequence diagram (sd).
    \item Page 278, section 9.3.7, difference between active and passive
        objects in sequence digrams.
    \item Page 279, list of sequence digram's "integration operators" e.g. alt,
        opt and loop.
    \item Page 286, section 9.6, timing diagrams, related to state machines
        (chapter 11).
    \item Page 292, chapter 10. How to specify operations with pre/post
        condtions and contracts.
    \item Page 354, section 12.5.1, list of qualities (quality attributes(?))
        with descriptions.
    \item Page 422, chapter 15 about Design patterns goes through singleton and
        composite.
\end{itemize}

%%%%%%%%%%%%%%%%%%%%%%%%%%%%%%%%%%%%%%%%%%%%%%%%%%%%%%%%%%%%%%%%%%%%%
% A Ravenscar-Compliant Run-Time Kernel for Safety Critical Systems
%%%%%%%%%%%%%%%%%%%%%%%%%%%%%%%%%%%%%%%%%%%%%%%%%%%%%%%%%%%%%%%%%%%%%
\section{A Ravenscar-Compliant Run-Time Kernel for Safety Critical Systems}
A good walkthrough about how a "naked" Run-Time Kernel is verified with UPPAAL.
Includes a good collection of references on Model-checking with automatons.

\begin{description}
    \item[Label:] lundqvist2003 \cite{lundqvist2003}
    \item[Date:] 2003
    \item[Abbrevations and terms:] None.
\end{description}

\begin{itemize}
    \item Page 8, section 4, some interesting references about timed automata
        and model-checking.
\end{itemize}

%%%%%%%%%%%%%%%%%%%%%%%%%%%%%%%%%%%%%%%%%%%%%%%%%%%%%%%%%%%%%%%%%%%%%
% Developing UPPAAL over 15 years
%%%%%%%%%%%%%%%%%%%%%%%%%%%%%%%%%%%%%%%%%%%%%%%%%%%%%%%%%%%%%%%%%%%%%
\section{Developing UPPAAL over 15 years}
Interesting paper about UPPAAL but not much value for my master thesis.

\begin{description}
    \item[Label:] behrmann2011 \cite{behrmann2011}
    \item[Date:] 2011
    \item[Abbrevations and terms:] None.
\end{description}

\begin{itemize}
    \item Page 5/137, first rule of survival "one should have a solid design
        and stick with it".
    \item Page 6/138, section "Tool building process", doxygen for comments,
        that's it.
    \item Page 6/138, intersting idea to use binary search to detect which
        addition of functionality introduced a newly detected bug.
    \item Page 7/139, Yahoo group for community communication.
\end{itemize}

%%%%%%%%%%%%%%%%%%%%%%%%%%%%%%%%%%%%%%%%%%%%%%%%%%%%%%%%%%%%%%%%%%%%%%
% Model Checking Applied to Embedded Software of University Satellite
%%%%%%%%%%%%%%%%%%%%%%%%%%%%%%%%%%%%%%%%%%%%%%%%%%%%%%%%%%%%%%%%%%%%%%
\section{Model Checking Applied to Embedded Software of University Satellite}

A good report with the basis in a master thesis about how to properly do
model-checking and why. This reports checks that the design of the system
works as intended and no deadlocks can occur. With the help of model-checking
some problems with the requirements specification were dedicted and fixed.

\begin{description}
    \item[Label:] alencar2013 \cite{alencar2013}
    \item[Year:] 2013
    \item[Abbrevations and terms:] None.
\end{description}

\begin{itemize}
    \item Page 2/127, section 2, good reasoning and listing of references.
\end{itemize}

%%%%%%%%%%%%%%%%%%%%%%%%%%%%%%%%%%%%%%%%%%%%%%%%%%%%%%%%%%%%%%%%%%%%%
% UPPAAL in a nutshell
%%%%%%%%%%%%%%%%%%%%%%%%%%%%%%%%%%%%%%%%%%%%%%%%%%%%%%%%%%%%%%%%%%%%%
\section{UPPAAL in a nutshell}
This report goes through the basics of UPPAAL on how to use it. It was written
while version 2.02 was the latest and now 4.X is available. Details are
therefore different when it comes to GUI and how to do different things within
UPPAAL but the main concepts remain the same.

\begin{description}
    \item[Label:] larsen1997 \cite{larsen1997}
    \item[Date:] 1997
    \item[Abbrevations and terms:] None.
\end{description}

\begin{itemize}
    \item In the abstract some terms are used that define the basics of UPPAAL.
        With UPPAAL one constructs a model, validate it with simulation and
        verify it with model-checking.
\end{itemize}

%%%%%%%%%%%%%%%%%%%%%%%%%%%%%%%%%%%%%%%%%%%%%%%%%%%%%%%%%%%%%%%%%%%%%
% Verification of real-time systems design
%%%%%%%%%%%%%%%%%%%%%%%%%%%%%%%%%%%%%%%%%%%%%%%%%%%%%%%%%%%%%%%%%%%%%
\section{Verification of real-time systems design}
Heavy on the theory but some RT-UML (Real-time UML) could perhaps be usefull in
the future.

\begin{description}
    \item[Label:] cambronero2010 \cite{cambronero2010}
    \item[Date:] 2010
    \item[Abbrevations and terms:] None.
\end{description}

\begin{itemize}
    \item Page 9, section 4.2; UML 2.0 Profile Schedulability, Performance and
        Time is mentioned.
    \item Page 8, section 3.1; Modelling, Validation and Verification. The
        Validation is done by model-checking in UPPAAL.
\end{itemize}

%%%%%%%%%%%%%%%%%%%%%%%%%%%%%%%%%%%%%%%%%%%%%%%%%%%%%%%%%%%%%%%%%%%%%
% Model checking of safety-critical software in the nuclear engineering domain
%%%%%%%%%%%%%%%%%%%%%%%%%%%%%%%%%%%%%%%%%%%%%%%%%%%%%%%%%%%%%%%%%%%%%
\section{Model checking of safety-critical software in the nuclear engineering
domain}
Some good thoughts on how to actually create the UPPAAL model.

\begin{description}
    \item[Label:] lahtinen2012 \cite{lahtinen2012}
    \item[Date:] 2012
    \item[Abbrevations and terms:] None.
\end{description}

\begin{itemize}
    \item Page 2/105, section 2, good descirption on how to do model checking.
    \item Page 9/111, section 5.3, good reasoning on why model-checking should
        be done.
\end{itemize}

%%%%%%%%%%%%%%%%%%%%%%%%%%%%%%%%%%%%%%%%%%%%%%%%%%%%%%%%%%%%%%%%%%%%%
% Formal Modeling and Verification of Operational Flight Program in a
% Small-Scale Unmanned Helicopter
%%%%%%%%%%%%%%%%%%%%%%%%%%%%%%%%%%%%%%%%%%%%%%%%%%%%%%%%%%%%%%%%%%%%%
\section{Formal Modeling and Verification of Operational Flight Program in a
Small-Scale Unmanned Helicopter}
This report has some interesting UPPAAL models over shared memory and mutex.

\begin{description}
    \item[Label:] lee2011 \cite{lee2011}
    \item[Date:] 2011
    \item[Abbrevations and terms:] None.
\end{description}

\begin{itemize}
    \item None.
\end{itemize}

%%%%%%%%%%%%%%%%%%%%%%%%%%%%%%%%%%%%%%%%%%%%%%%%%%%%%%%%%%%%%%%%%%%%%
% Modeling a Spacewire architecture using timed automata to compute
% worst-case end-to-end delays.
%%%%%%%%%%%%%%%%%%%%%%%%%%%%%%%%%%%%%%%%%%%%%%%%%%%%%%%%%%%%%%%%%%%%%
\section{Modeling a Spacewire architecture using timed automata to compute
worst-case end-to-end delays}
Some ideas I got when reading this paper was that some upper bound can perhaps
be verified when it comes to some time property. Also the model should check
for starvation of any of the queues as well as worst case time of any queue.

\begin{description}
    \item[Label:] lee2011 \cite{lee2011}
    \item[Date:] 2011
    \item[Abbrevations and terms:] None.
\end{description}

\begin{itemize}
    \item Page 1, section 1, interesting idea to compute worst-case
        endpoint-to-endpoint in spacewire.
\end{itemize}

%%%%%%%%%%%%%%%%%%%%%%%%%%%%%%%%%%%%%%%%%%%%%%%%%%%%%%%%%%%%%%%%%%%%%%
% State-of-the-art Tools and Techniques for Quantitative Modeling and Analysis
% of Embedded Systems
%%%%%%%%%%%%%%%%%%%%%%%%%%%%%%%%%%%%%%%%%%%%%%%%%%%%%%%%%%%%%%%%%%%%%%
\section{State-of-the-art Tools and Techniques for Quantitative Modeling and
Analysis of Embedded Systems}
This report has a simple UPPAAL model about a train crossing. It also has
some reasoning about model based development.

\begin{description}
    \item[Label:] bozga2012 \cite{bozga2012}
    \item[Year:] 2012
    \item[Abbrevations and terms:] None.
\end{description}

\begin{itemize}
    \item None.
\end{itemize}

%%%%%%%%%%%%%%%%%%%%%%%%%%%%%%%%%%%%%%%%%%%%%%%%%%%%%%%%%%%%%%%%%%%%%%
% Case study on distributed and fault tolerant system modeling based on timed
% automata
%%%%%%%%%%%%%%%%%%%%%%%%%%%%%%%%%%%%%%%%%%%%%%%%%%%%%%%%%%%%%%%%%%%%%%
\section{Case study on distributed and fault tolerant system modeling based on
timed automata}
This is one of the better reports found so far. Provides a good background for
the report and throughout the report goes through the reasoning behind the
design choices made when creating the UPPAAL models.

\begin{description}
    \item[Label:] waszniowski2009 \cite{waszniowski2009}
    \item[Year:] 2009
    \item[Abbrevations and terms:] None.
\end{description}

\begin{itemize}
    \item Page 2/1679, section 1 \& 2, really interesting discussion of related
        work. Some good references concerning CAN, embedded systems, time and
        event triggered systems.
    \item Page 6/1683, section 4.2, Modelling of multitasking.
    \item Page 12/1689, section 5, definition of properties to be verified
        "bounded liveness".
    \item Page 13/1690, section 5, no deadlock.
    \item Page 13/1690, section 5, what is non-zenoness?
\end{itemize}

%%%%%%%%%%%%%%%%%%%%%%%%%%%%%%%%%%%%%%%%%%%%%%%%%%%%%%%%%%%%%%%%%%%%%%
% Modeling and Analysis of Radiation therapy system with respiratory
% compensation using UPPAAL.
%%%%%%%%%%%%%%%%%%%%%%%%%%%%%%%%%%%%%%%%%%%%%%%%%%%%%%%%%%%%%%%%%%%%%%
\section{Modeling and Analysis of Radiation therapy system with respiratory
compensation using UPPAAL}
This report includes a simple UPPAAL model that could be tested to understand
UPPAAL a bit more.

\begin{description}
    \item[Label:] man2011 \cite{man2011}
    \item[Year:] 2011
    \item[Abbrevations and terms:] None.
\end{description}

\begin{itemize}
    \item Page 3/52, section 3, short explanation of UPPAAL computation tree
    logic (CTL) with references (A[], A\textless\textgreater,
    \textless\textgreater, E\textless\textgreater, deadlock, P.state).
\end{itemize}

%%%%%%%%%%%%%%%%%%%%%%%%%%%%%%%%%%%%%%%%%%%%%%%%%%%%%%%%%%%%%%%%%%%%%%
% A theory of timed automata
%%%%%%%%%%%%%%%%%%%%%%%%%%%%%%%%%%%%%%%%%%%%%%%%%%%%%%%%%%%%%%%%%%%%%%
\section{A theory of timed automata}
Heavy on the math and logic. This is the original report on timed automata
theory with over 5500 citings according to google scholar.

\begin{description}
    \item[Label:] alur1994 \cite{alur1994}
    \item[Year:] 1994
    \item[Abbrevations and terms:] None.
\end{description}

\begin{itemize}
    \item None.
\end{itemize}
%%%%%%%%%%%%%%%%%%%%%%%%%%%%%%%%%%%%%%%%%%%%%%%%
% New reference
%%%%%%%%%%%%%%%%%%%%%%%%%%%%%%%%%%%%%%%%%%%%%%%%
%\section{Title}
%
% Some thoughts about the reference here.
%
%\begin{description}
%    \item[Label:] web:ada-high-integrity \cite{web:ada-high-integrity}
%    \item[Date:] MM-YYYY
%    \item[Abbrevations and terms:]
%       ASIS,\nomenclature{\textbf{ASIS}}{\textbf{Ada Semantic Interface
%       Specification} provides a standard mechanism for obtaining information
%       about an Ada program or its components. ASIS is an ISO standard.}
%\end{description}
%
%
% \begin{itemize}
%     \item Page 34, section 5.11 has valuable information about access types and
%     \item Page 35, good reasoning about the use of exceptions.
%     \item Page 37, section 5.13 has some good notes on Tasking in
% \end{itemize}
