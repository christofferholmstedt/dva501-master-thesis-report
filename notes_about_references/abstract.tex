\chapter*{Abstract}
\thispagestyle{empty} % No page numbering on this page
Speech recognition has been an important area of research during the past
decades.  The usage of automatic speech recognition systems is rapidly
increasing among different areas, such as mobile telephony, automotive,
healthcare, robotics and more. However, despite the existence of many speech
recognition systems, most of them use platform specific and non-publicly
available software. Nevertheless, it is possible to develop speech recognition
systems using already existing open source technology.

The aim of this master’s thesis is to develop an interactive and speaker
independent speech recognition system. The system shall be able to identify
predetermined keywords from incoming live speech and in response, play audio
files with related information.  Moreover, the system shall be able to provide
a response even if no keyword was identified. For this project, the system was
implemented using PocketSphinx, a speech recognition library, part of the open
source Sphinx technology by the Carnegie Mellon University.

During the implementation of this project, the automation of different steps of
the process, was a key factor for a successful completion. This automation
consisted on the development of different tools for the creation of the
language model and the dictionary, two important components of the system.
Similarly, the audio files to be played after identifying a keyword, as well as
the evaluation of the system’s performance, were fully automated.

The tests run show encouraging results and demonstrate that the system is a
feasible solution that could be implemented and tested in a real embedded
application. Despite the good results, possible improvements can be
implemented,

% Lorem ipsum dolor sit amet, consectetur adipiscing elit. Sed nec ligula vel
% ante placerat dapibus. Vestibulum non sollicitudin tellus. Sed varius
% vestibulum libero, in euismod purus dapibus sed. Pellentesque malesuada,
% urna nec lobortis egestas, felis elit varius ante, in tincidunt nunc metus
% id diam. Morbi sagittis velit at ipsum sodales, in tempor nulla volutpat.
% Integer a neque eros. In ipsum erat, pellentesque at dapibus eget, pretium
% sed risus. Vivamus quis metus vitae felis tempus vulputate eget id lacus.
% Aenean sit amet gravida quam, et ornare erat. Praesent ullamcorper mattis
% dolor, non interdum orci aliquam a. In tincidunt semper lectus. Duis quis
% mi ac felis vehicula mollis. Fusce molestie arcu urna, a tempor turpis
% mattis nec. Aenean in massa a risus euismod aliquet vel non libero.
% 
% Aenean luctus quam ut neque viverra, quis vulputate mi hendrerit. Nunc id leo
% id dui semper blandit. Nulla orci ligula, posuere non lacinia non, aliquet non
% massa. Quisque gravida in mi eget volutpat. Fusce convallis felis sed dolor
% ornare rhoncus. Nullam sed lacus id lectus venenatis pulvinar eget ac arcu. Sed
% id turpis in leo sollicitudin euismod nec ac enim. Nunc vitae lacus lectus.
% Class aptent taciti sociosqu ad litora torquent per conubia nostra, per
% inceptos himenaeos. Integer gravida, orci vitae viverra ullamcorper, quam
% libero tempor est, non gravida magna eros id augue. In lobortis et est auctor
% congue. Aliquam sed ante lacus. Vestibulum aliquam elit eget elementum
% facilisis. Quisque eget felis eu leo pharetra pharetra.
% 
% Phasellus nec erat vehicula, ornare urna sit amet, dictum metus. Duis ac
% tristique enim. Maecenas malesuada neque non libero egestas, ut tempor justo
% imperdiet. Praesent et arcu id magna porta dictum eget sit amet dolor.
% Suspendisse porta eros at eleifend hendrerit. Nunc neque elit, hendrerit eu
% odio sed, semper laoreet est. Aliquam interdum erat vel tortor luctus
% adipiscing. Curabitur dictum nulla sit amet elit malesuada, vitae elementum
% dolor malesuada. Vivamus quis tellus quis ligula fringilla commodo non a justo.
% Vestibulum egestas mattis mi, eu volutpat mi mollis vitae. Proin quis urna id
% diam lobortis rutrum. Etiam ultricies consequat ipsum eget congue.

% Keywords
\textbf{Keywords:} Ada Tasks, Formal Methods, UPPAAL
