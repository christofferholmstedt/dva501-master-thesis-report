\chapter{Space plug-and-play Architecture}

%%%%%%%%%%%%%%%%%%%%%%%%%%%%%%%%%%%%%%%%%%%%%%%%%%%%%%%%%%%
% AFRL Plug-and-Play Spacecraft Avionics Experiement (SAE)
%%%%%%%%%%%%%%%%%%%%%%%%%%%%%%%%%%%%%%%%%%%%%%%%%%%%%%%%%%%
\section{AFRL Plug-and-Play Spacecraft Avionics Experiement (SAE)}

First experiment with SPA on-orbit and describes the benefits of the
plug-and-play architecture. "The benefits of the modular, reconfigurable SPA
interface for cost and schedule savings were demonstrated on numerous occasions
during integration [...]".

\begin{description}
    \item[Label:] martin2012 \cite{martin2012}
    \item[Year:] 2012
    \item[Abbrevations and terms:] None.
\end{description}

\begin{itemize}
    \item Page 1, section 1, rename of Space Plug-and-play Avionics to Space
        Plug-and-play Architecture.
    \item Page 1, section 1, mentions SPA over USB, SpaceWire, SPA-1 (I2C) as
        well as Ethernet.
\end{itemize}

%%%%%%%%%%%%%%%%%%%%%%%%%%%%%%%%%%%%%%%%%%%%%%%%%%%%%%%%%%%%%%%%%%%%%%
% Developing an Ontology for Standardizing Space Systems Data Exchange
%%%%%%%%%%%%%%%%%%%%%%%%%%%%%%%%%%%%%%%%%%%%%%%%%%%%%%%%%%%%%%%%%%%%%%
\section{Developing an Ontology for Standardizing Space Systems Data Exchange}

This goes through the creation of an ontology, common data dictionary and
taxonomies. Defines taxonomies as categories in an ontology.

The difference between services and data streams should be mentioned and in my
report the scope is only on data streams (sensors), in the area of Internet of
Things.

\begin{description}
    \item[Label:] hansen2012 \cite{hansen2012}
    \item[Year:] 2012
    \item[Abbrevations and terms:] None.
\end{description}

\begin{itemize}
    \item Page 2, section 3, the difference between services (often a-periodic)
        and data streams (often periodic).
    \item Page 2, section 3, mentions that the ontology/taxonomies should be
        applicable for any type of unmanned system.
    \item Page 4, section 3, figure 3, 4 and 5 shows taxonomies and secondary
        taxonomies.
\end{itemize}
%%%%%%%%%%%%%%%%%%%%%%%%%%%%%%%%%%%%%%%%%%%%%%%%%%%%%%%
% Space Plug-and-Play Architecture - Logical Interface
%%%%%%%%%%%%%%%%%%%%%%%%%%%%%%%%%%%%%%%%%%%%%%%%%%%%%%%
\section{SPA - Logical Interface}

Basic overview of the SPA middleware and the specifics on SPA headers.

\begin{description}
    \item[Label:] spa:logical-interface \cite{spa:logical-interface}
    \item[Year:] YYYY
    \item[Abbrevations and terms:]
        AIAA,\nomenclature{\textbf{AIAA}}{\textbf{American Institute of
            Aeronatuics and Astronautics}}
        CAS,\nomenclature{\textbf{CAS}}{\textbf{Central Addressing
            Service} is responsible for providing logical address blocks to be
            assigned to each hardware or software component. The CAS stores the
            logical address block and logical address for each SPA Manager in
            the SPA Network.}
        Component, A SPA \nomenclature{\textbf{Component}}{\textbf{Component}
            is an endpoint whose interface conforms to the SPA interface
            standard, and does not connect to another SPA object via a
            different port.}
        CUUID,\nomenclature{\textbf{CUUID}}{\textbf{Component Universally
            Unique Identifier}}
        SM-x,\nomenclature{\textbf{SM-x}}{\textbf{SPA Subnet Manager}, where x
            represents a given technology protocol.}
        SPA,\nomenclature{\textbf{SPA}}{\textbf{Space Plug-n-Play
            Architecture}}
        URN,\nomenclature{\textbf{URN}}{\textbf{Uniform Resource Name}}
        UUID,\nomenclature{\textbf{UUID}}{\textbf{Universially Unique
            Identifier}}
        xTEDS,\nomenclature{\textbf{xTEDS}}{\textbf{Extensible Transducer
            Electronic Data Sheet}}
        XUUID,\nomenclature{\textbf{XUUID}}{\textbf{xTEDS Universially Unique
            Identifier}}
\end{description}

\begin{itemize}
    \item Page 8, SPA Guidebook has some good examples about how SPA messages
        should be constructed.
    \item Page 13, [SPA-LOG-0005] Guaranteed Delivery Extended Header.
    \item Page 14, [SPA-LOG-0007] SPA maximum size 65536 bytes.
    \item Page 16, [SPA-LOG-0010] 16 bit checksum in each SPA message as in TCP
        RFC 793.
    \item Page 53, brief network discovery description.
    \item Page 64, Component Identification RFC 4122 reference.
\end{itemize}

%%%%%%%%%%%%%%%%%%%%%%%%%%%%%%%%%%%%%%%%%%%%%%%%%%%%%%%%%%%%%
% Space Plug-and-Play Architecture - Local Subnet Adaptation
%%%%%%%%%%%%%%%%%%%%%%%%%%%%%%%%%%%%%%%%%%%%%%%%%%%%%%%%%%%%%
\section{SPA - Local Subnet Adaptation}

This standard has a key part in my work.

\begin{description}
    \item[Label:] MISSING REFERENCE % spa:logical-interface \cite{spa:logical-interface}
    \item[Year:] YYYY
    \item[Abbrevations and terms:]
        EP,\nomenclature{\textbf{EP}}{SPA \textbf{Endpoint}}
        IPC,\nomenclature{\textbf{IPC}}{\textbf{Inter-process communication}}
        LS,\nomenclature{\textbf{LS}}{SPA \textbf{Lookup Service}}
        SM-L,\nomenclature{\textbf{SM-L}}{\textbf{SPA Local Interconnect}}
        XML,\nomenclature{\textbf{XML}}{\textbf{Extensible Markup Language}}
        QoS,\nomenclature{\textbf{QoS}}{\textbf{Quality of Service}}
\end{description}

\begin{itemize}
    \item Page 11, references to UDP/IP when it comes to IPC.
    \item Page 13, section 6, discovery concerns which is a key part of a SPA
        local subnet.
    \item Page 14, section 8, any implementation that ignores QoS is SPA
        compliant.
    \item Page 15, section 9.2, subnet managers should be able to detect when
        services in the local subnet are not available anymore.
    \item Page 15-16, what's the difference between a Subnet Manager and Subnet
        Manager Gateway?
    \item Page 19, section 11.1, What is the rationale for UDP/IP as IPC?
\end{itemize}

%%%%%%%%%%%%%%%%%%%%%%%%%%%%%%%%%%%%%%%%%%%%%%%%%%%%%%%%%%%%%
% Space Plug-and-Play Architecture - Networking
%%%%%%%%%%%%%%%%%%%%%%%%%%%%%%%%%%%%%%%%%%%%%%%%%%%%%%%%%%%%%
\section{SPA - Networking}

This brings up the details about bootstraping the network. This report is a key
part of my master thesis.

\begin{description}
    \item[Label:] spa:networking \cite{spa:networking}
    \item[Year:] YYYY
    \item[Abbrevations and terms:]
        MTU,\nomenclature{\textbf{MTU}}{\textbf{Maximum Transmission Unit}}
        EOP,\nomenclature{\textbf{EOP}}{\textbf{End of Packet}}
\end{description}

\begin{itemize}
    \item Page 11, references to UDP/IP when it comes to IPC.
    \item Page 2, only one active CAS per SPA network.
    \item Page 2, CAS will exists on a SPA Local Subnet.
    \item Page 3, only one active LS per SPA network.
    \item Page 4, section 4.2.1, the rationale behind the small 32 bit logical
        address.
    \item Page 7-9, section 4.2.1.3, what kind of routing information should be
        stored throughout the network.
    \item Page 9, a SM is the only component that will store none application
        layer information such as IP-addresses and other underlying technology
        specifics.
    \item Page 14, section 6.1, how multiple SM-x on the same subnet cooperate
        (important).
    \item Page 15, [SPA-NET-0005], all SM-x in the same network should be able
        to communicate directly with each other.
    \item Page 17, [SPA-NET-0026], a SM-x that just recieved a logical address
        space should request more address spaces for other SM-x on its
        off-side.
    \item Page 18, section 7.4.3(g), some reasoning about lossy networks and
        "guaranteed delivery" flag.
\end{itemize}

%%%%%%%%%%%%%%%%%%%%%%%%%%%%%%%%%%%%%%%%%%%%%%%%%%%%%%%%%%%%%
% Space Plug-and-Play Architecture - xTEDS and Ontology
%%%%%%%%%%%%%%%%%%%%%%%%%%%%%%%%%%%%%%%%%%%%%%%%%%%%%%%%%%%%%
\section{SPA - xTEDS and Ontology}

This is pretty heavy to read, a lot of details. Read this again when developing
the XML parts of the application.

\begin{description}
    \item[Label:] MISSING REFERENCE % spa:logical-interface \cite{spa:logical-interface}
    \item[Year:] YYYY
    \item[Abbrevations and terms:]
        XSD,\nomenclature{\textbf{XSD}}{\textbf{XML Schema Definition}}
\end{description}

\begin{itemize}
    \item Page 2, a few sections on how xTEDS are validated.
    \item Page 5, mixed case is used instead of Ada underline.
\end{itemize}

%%%%%%%%%%%%%%%%%%%%%%%%%%%%%%%%%%%%%%%%%%%%%%%%
% New reference
%%%%%%%%%%%%%%%%%%%%%%%%%%%%%%%%%%%%%%%%%%%%%%%%
%\section{Title}
%
% Some thoughts about the reference here.
%
%\begin{description}
%    \item[Label:] web:ada-high-integrity \cite{web:ada-high-integrity}
%    \item[Date:] MM-YYYY
%    \item[Abbrevations and terms:]
%       ASIS,\nomenclature{\textbf{ASIS}}{\textbf{Ada Semantic Interface
%       Specification} provides a standard mechanism for obtaining information
%       about an Ada program or its components. ASIS is an ISO standard.}
%\end{description}
%
%
% \begin{itemize}
%     \item Page 34, section 5.11 has valuable information about access types and
%     \item Page 35, good reasoning about the use of exceptions.
%     \item Page 37, section 5.13 has some good notes on Tasking in
% \end{itemize}
