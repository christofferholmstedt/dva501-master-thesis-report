\chapter{Space plug-and-play Avionics}

%%%%%%%%%%%%%%%%%%%%%%%%%%%%%%%%%%%%%%%%%%%%%%%%%%%%%%%
% Space Plug-and-Play Architecture - Logical Interface
%%%%%%%%%%%%%%%%%%%%%%%%%%%%%%%%%%%%%%%%%%%%%%%%%%%%%%%
\section{Space Plug-and-Play Architecture - Logical Interface}

Basic overview of the SPA middleware and the specifics on SPA headers.

\begin{description}
    \item[Label:] spa:logical-interface \cite{spa:logical-interface}
    \item[Year:] XXXX
    \item[Abbrevations and terms:]
        AIAA,\nomenclature{\textbf{AIAA}}{\textbf{American Institute of
            Aeronatuics and Astronautics}}
        CAS,\nomenclature{\textbf{CAS}}{\textbf{Central Addressing
            Service} is responsible for providing logical address blocks to be
            assigned to each hardware or software component. The CAS stores the
            logical address block and logical address for each SPA Manager in
            the SPA Network.}
        CUUID,\nomenclature{\textbf{CUUID}}{\textbf{Component Universally
            Unique Identifier}}
        SM-x,\nomenclature{\textbf{SM-x}}{\textbf{SPA Subnet Manager}, where x
            represents a given technology protocol.}
        SPA,\nomenclature{\textbf{SPA}}{\textbf{Space Plug-n-Play
            Architecture}}
        URN,\nomenclature{\textbf{URN}}{\textbf{Uniform Resource Name}}
        UUID,\nomenclature{\textbf{UUID}}{\textbf{Universially Unique
            Identifier}}
        xTEDS,\nomenclature{\textbf{xTEDS}}{\textbf{Extensible Transducer
            Electronic Data Sheet}}
        XUUID,\nomenclature{\textbf{XUUID}}{\textbf{xTEDS Universially Unique
            Identifier}}
\end{description}


\begin{itemize}
    \item Page 8, SPA Guidebook has some good examples about how SPA messages
        should be constructed.
    \item Page 13, [SPA-LOG-0005] Guaranteed Delivery Extended Header.
    \item Page 14, [SPA-LOG-0007] SPA maximum size 65536 bytes.
    \item Page 16, [SPA-LOG-0010] 16 bit checksum in each SPA message as in TCP
        RFC 793.
    \item Page 53, brief network discovery description.
    \item Page 64, Component Identification RFC 4122 reference.
\end{itemize}
%%%%%%%%%%%%%%%%%%%%%%%%%%%%%%%%%%%%%%%%%%%%%%%%
% New reference
%%%%%%%%%%%%%%%%%%%%%%%%%%%%%%%%%%%%%%%%%%%%%%%%
%\section{Title}
%
% Some thoughts about the reference here.
%
%\begin{description}
%    \item[Label:] web:ada-high-integrity \cite{web:ada-high-integrity}
%    \item[Date:] MM-YYYY
%    \item[Abbrevations and terms:]
%       ASIS,\nomenclature{\textbf{ASIS}}{\textbf{Ada Semantic Interface
%       Specification} provides a standard mechanism for obtaining information
%       about an Ada program or its components. ASIS is an ISO standard.}
%\end{description}
%
%
% \begin{itemize}
%     \item Page 34, section 5.11 has valuable information about access types and
%     \item Page 35, good reasoning about the use of exceptions.
%     \item Page 37, section 5.13 has some good notes on Tasking in
% \end{itemize}
